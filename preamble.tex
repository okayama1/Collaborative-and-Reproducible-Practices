\usepackage{booktabs}
\usepackage{colortbl}
\usepackage{graphics} 
\usepackage{float}


%%%%%%%%%%%% 
%\usepackage{bbm}
%\usepackage{threeparttable}
%\usepackage{expex}

\usepackage{fancyhdr}
\usepackage[singlelinecheck=false]{caption}
\usepackage{tabularx}


\setlength{\headheight}{60pt}

\pagestyle{fancy}



\rhead{\includegraphics[height=60pt]{AIMSLogoinline.png}}

\lhead{}

\usepackage[a4paper, total={6in, 8in}]{geometry}
\renewcommand{\headrulewidth}{0.7pt}


\definecolor{aimsblue}{rgb}{0, 0.29, 0.55}

\newgeometry{textwidth=15cm,textheight=18cm}
\renewcommand{\headrule}{{\color{aimsblue}%
 \hrule width\headwidth height\headrulewidth \vskip-\headrulewidth}}

\usepackage{xcolor}
\usepackage{sectsty}
\sectionfont{\color{aimsblue}}  
\subsectionfont{\color{aimsblue}} 
\subsubsectionfont{\color{aimsblue}} 

% Length to control the \fancyheadoffset and the calculation of \headline
% simultaneously
\newlength\FHoffset
\setlength\FHoffset{1cm}

\addtolength\headwidth{2\FHoffset}
\fancyheadoffset{\FHoffset}

% these lengths will control the headrule trimming to the left and right 
\newlength\FHleft
\newlength\FHright

% here the trimmings are controlled by the user
\setlength\FHleft{1cm}
\setlength\FHright{0cm}

% The new definition of headrule that will take into acount the trimming(s)
\newbox\FHline
\setbox\FHline=\hbox{\hsize=\paperwidth%
 \hspace*{\FHleft}%
 \rule{\dimexpr\headwidth-\FHleft-\FHright\relax}{\headrulewidth}\hspace*{\FHright}%
}